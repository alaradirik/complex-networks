\documentclass{article}
\usepackage[utf8]{inputenc}
\usepackage[english]{babel}
\usepackage[]{amsthm} 
\usepackage[]{amssymb} 
\usepackage{amsmath}
\usepackage[nottoc]{tocbibind}

\DeclareRobustCommand{\divby}{%
  \mathrel{\text{\vbox{\baselineskip.65ex\lineskiplimit0pt\hbox{.}\hbox{.}\hbox{.}}}}%
}
\title{Midterm 1 - Q4}
\author{Alara Dirik}
\date\today

\begin{document}
\maketitle 

\subsection*{Problem}
The "Finding and Evaluating Community Structure in Networks" paper by Newman and Girvan {\bf \cite{Newman2004FindingAE}} introduces modularity Q for the first time, however the description of Q in the paper is not clear. Do some research and then describe it in a better way.


\subsection*{Answer}

The paper by Newman and Girvan uses modularity Q as a metric to determine the number of clusters in a given network. Given a network divided into $\ell$ communities $\left\{c_{1}, \ldots, c_{\ell}\right\}$, the modularity of this parrticular division is the number of edges within the communities minus the expected number of edges within the communities if the edges were distributed at random {\bf \cite{Newman2016CommunityDI}}. \newline

\noindent
Modularity can be formulated as 
$Q=\frac{1}{2 m} \sum_{i j}\left[A_{i j}-\frac{k_{i} k_{j}}{2 m}\right] \delta\left(c_{i}, c_{j}\right)$ where,

\begin{itemize}
    \item $m$ is the number of edges
    \item $A$ is the adjacency matrix
    \item $k_{i}$ is the degree of node $i$
    \item $c_{i}$ is the community ID of node $i$
\end{itemize}

\noindent
Intuitively, modularity Q lies between $[-1, 1]$ and is positive if the number of edges within groups exceeds the expected number edges. $0.3 < Q < 0.7$ denotes significant community structure, thus modularity Q can be directly optimized to detect communities in a network.

\nocite{*}
\bibliographystyle{unsrt}
\bibliography{references}
\end{document}